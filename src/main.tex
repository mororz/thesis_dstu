\documentclass[a4paper,14pt]{extreport}

% ----------------------- Font settings --------------------%

\usepackage[english,ukrainian]{babel}
\usepackage{fontspec}
\setmainfont{Times New Roman}
\setsansfont{Comic Sans MS}
\setmonofont{Courier New}

% ----------------------------------------------------------%

\usepackage{extsizes} % allows 14pt
\usepackage{indentfirst} % always indent new paragraph
\linespread{1.3}

% ------------------------- Margins ------------------------%

\usepackage{geometry}
\geometry{left=2.5cm}
\geometry{right=1.5cm}
\geometry{top=2cm}
\geometry{bottom=2cm}

\setlength{\parindent}{2.5em} % dstu demands 5 indentation to be 5 symbols

% ----------------------------------------------------------%

% -----------------Sections. New and improved---------------%

\usepackage{titlesec}

\titleformat{\chapter}[display]
        {\filcenter}
        {\MakeUppercase{\chaptertitlename} \thechapter}
        {8pt}
        { \MakeUppercase}{}

\titlespacing*{\chapter}{0pt}{-25pt}{8pt}

% ----------------------------------------------------------%

% ------------------------Headers---------------------------%

\usepackage{fancyhdr}
\pagestyle{fancy}
\fancyhf{}
\fancyhead[R]{\thepage}
\fancyheadoffset{0mm}
\fancyfootoffset{0mm}
\setlength{\headheight}{16pt}
\renewcommand{\headrulewidth}{0pt}
\renewcommand{\footrulewidth}{0pt}
\fancypagestyle{plain}{ 
	\fancyhf{}
	\rhead{\thepage}}


% ----------------------------------------------------------%


\begin{document}

\chapter{Розділ перший}

\section{Перший підрозділ першого розділу}

Яків до створення штучного інтелекту. Однією з задач цієї сфери є класифікація
текстів. Класифікація текстів широко використовується у таких областях
як виявлення спаму, аналіз тональності та в діалогових системах. Технології
на базі NLP стають дедалі ширшим. Наприклад, телефони уже підтримують
інтелектуальне розпізнавання голосу та рукописного тексту; пошукові систе-
ми надають доступ до інформації, закодованої в неструктурованому тексті;
машинний переклад дозволяє нам отримувати тексти, написані китайською
мовою, і читати їх іспанською мовою.

Метою класифікації текстів є встановлення відповідності між документами
(таких як електронні листи, повідомлення, огляди продуктів тощо) та
деякими категоріями. Такими категоріями можуть бути відгуки про результати,
спам/не спам або мова, в якій був набраний документ. В даний час
домінуючим підходом до побудови таких класифікаторів є машинне навчання,
тобто навчання моделі на основі багатьох прикладів, а не програмування.

\tableofcontents

\end{document}
